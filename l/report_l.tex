%----------------------------------------------------------------------------------------
%	PACKAGES AND DOCUMENT CONFIGURATIONS
%----------------------------------------------------------------------------------------

\documentclass{article}

\usepackage{natbib} % Required to change bibliography style to APA
\usepackage{amsmath} % Required for some math elements 
\usepackage{listings}
\usepackage{color}
\usepackage{circuitikz}
\usetikzlibrary{circuits.logic.IEC,calc}

\definecolor{dkgreen}{rgb}{0,0.6,0}
\definecolor{gray}{rgb}{0.5,0.5,0.5}
\definecolor{mauve}{rgb}{0.58,0,0.82}

\lstset{frame=tb,
  language=Verilog,
  aboveskip=3mm,
  belowskip=3mm,
  showstringspaces=false,
  columns=flexible,
  basicstyle={\small\ttfamily},
  numbers=none,
  numberstyle=\tiny\color{gray},
  captionpos=b,                   % sets the caption-position to bottom
  title=\lstname,                 % show the filename of files included with \lstinputlisting;
  keywordstyle=\color{blue},
  commentstyle=\color{dkgreen},
  stringstyle=\color{mauve},
  breaklines=true,
  breakatwhitespace=true,
  tabsize=4
}

\newenvironment{statement}{\par\vspace{50ex}}{\clearpage}

\setlength\parindent{0pt} % Removes all indentation from paragraphs

% Make numbering in the enumerate environment by letter rather than number (e.g. section 6)
\renewcommand{\labelenumi}{\alph{enumi}.}

%----------------------------------------------------------------------------------------
%	DOCUMENT INFORMATION
%----------------------------------------------------------------------------------------

\title{\textsc{Lab L} \\ Hardware Building Blocks } % Title

\author{\textsc{Mohammadamin Bandali}} % Author name

\date{} % no date

\begin{document}

\maketitle % Insert the title, author and date

\begin{center}
\begin{tabular}{l r}
Student Number: & XXXXXXXXX \\ 
Date Performed: & March 31, 2015 \\ % Date the experiment was performed
Lab Location: & Lassonde 1006 \\ 
Course Name: & Computer Organization \\ 
Course Code: & EECS 2021 \\ 
Course Section: & Z, Lab-02\\ 
Instructor: & Professor Peter Lian % Instructor/supervisor
\end{tabular}
\end{center}

\begin{statement}
“The work in this report is my own. I have read and understood York University
academic honesty guidelines and I did not violate Senate Policy on Academic
Honesty in writing this report.”
\end{statement}

%----------------------------------------------------------------------------------------
%	SECTION 1
%----------------------------------------------------------------------------------------

\section{Introduction}

The purpose of this lab is gaining more familiarity with the Verilog language
and using it to build and use memory-less and combinational components.
Throughout the lab the student is asked to:

\begin{itemize}
\item use abstraction to encapsulate a circuit in form of a separate module,

\item enhance an existing 2-to-1 multiplexer to handle 2-bit busses instead
of 1-bit wires and then extend it to 32-bits,

\item design a 4-to-1 multiplexer,

\item encapsulate a simple full adder as a module and use it to make a
32-bit full adder,

\item extend the full adder to perform subtraction as well,
and thus have made a simple arithmetic unit,

\item reading user input from the standard input, and

\item finally, use all the previous pieces to build an ALU with
logical \verb$AND$/\verb$OR$ and addition/subtraction and \verb$slt$ operations
along with a \verb$zero$ flag exception.

\end{itemize}

%----------------------------------------------------------------------------------------
%	SECTION 2
%----------------------------------------------------------------------------------------

\section{Methods and Equipment}

\begin{center}
\begin{tabular}{l r}
Computer: & MacBook Air 13-inch, Mid 2013 \\
CPU: & 1.7GHz dual-core Intel Core i7 \\
RAM: & 8GB 1600MHz LPDDR3 \\
Operating System: & Arch GNU/Linux \\
Kernel Version: & 3.17.6
\end{tabular}
\end{center}

%----------------------------------------------------------------------------------------
%	SECTION 3
%----------------------------------------------------------------------------------------

\section{Experimental Procedure and Results}

\subsection{LabL1}
\begin{enumerate}
\item[4. ] The mux behaves as expected:
\begin{verbatim}
PASS: a=0 b=0 c=0 z=0
PASS: a=0 b=0 c=1 z=0
PASS: a=0 b=1 c=0 z=0
PASS: a=0 b=1 c=1 z=1
PASS: a=1 b=0 c=0 z=1
PASS: a=1 b=0 c=1 z=0
PASS: a=1 b=1 c=0 z=1
PASS: a=1 b=1 c=1 z=1
\end{verbatim}

  All tests pass and the mux is functioning properly.

\end{enumerate}

\subsection{LabL2}
\begin{enumerate}
\item[6. ] Having a multiplexer for 2-bit input/output is like having two
multiplexers that each handle one bit of input/output. The two multiplexers
share the same control bit. The shown circuit for \verb$yMux2$ is a schema
for the given description of such multiplexer.

\item[9. ] The enhanced (2-bit) mux (using \verb$yMux2$) behaves as expected:
\begin{verbatim}
PASS: a=00 b=00 c=0 z=00
PASS: a=00 b=00 c=1 z=00
PASS: a=00 b=01 c=0 z=00
PASS: a=00 b=01 c=1 z=01
PASS: a=00 b=10 c=0 z=00
PASS: a=00 b=10 c=1 z=10
PASS: a=00 b=11 c=0 z=00
PASS: a=00 b=11 c=1 z=11
PASS: a=01 b=00 c=0 z=01
PASS: a=01 b=00 c=1 z=00
PASS: a=01 b=01 c=0 z=01
PASS: a=01 b=01 c=1 z=01
PASS: a=01 b=10 c=0 z=01
PASS: a=01 b=10 c=1 z=10
PASS: a=01 b=11 c=0 z=01
PASS: a=01 b=11 c=1 z=11
PASS: a=10 b=00 c=0 z=10
PASS: a=10 b=00 c=1 z=00
PASS: a=10 b=01 c=0 z=10
PASS: a=10 b=01 c=1 z=01
PASS: a=10 b=10 c=0 z=10
PASS: a=10 b=10 c=1 z=10
PASS: a=10 b=11 c=0 z=10
PASS: a=10 b=11 c=1 z=11
PASS: a=11 b=00 c=0 z=11
PASS: a=11 b=00 c=1 z=00
PASS: a=11 b=01 c=0 z=11
PASS: a=11 b=01 c=1 z=01
PASS: a=11 b=10 c=0 z=11
PASS: a=11 b=10 c=1 z=10
PASS: a=11 b=11 c=0 z=11
PASS: a=11 b=11 c=1 z=11
\end{verbatim}

  All tests still pass and the 2-bit mux is working correctly.

\end{enumerate}

\subsection{LabL3}
\begin{enumerate}
\item[14. ] The behaviour using a 32-bit width bus is exactly the same
(same output as LabL2). However, the binary representations of the outputs
are now 32-bit wide so I haven't included them.

\item[17. ] All tests pass with randomly generated values for \verb$a$,
\verb$b$ and \verb$c$:
\begin{verbatim}
PASS: a=00010010000101010011010100100100
      b=11000000100010010101111010000001
      c=1 z=11000000100010010101111010000001
PASS: a=10110001111100000101011001100011
      b=00000110101110010111101100001101
      c=1 z=00000110101110010111101100001101
PASS: a=10110010110000101000010001100101
      b=10001001001101110101001000010010
      c=1 z=10001001001101110101001000010010
PASS: a=00000110110101111100110100001101
      b=00111011001000111111000101110110
      c=1 z=00111011001000111111000101110110
PASS: a=01110110110101000101011111101101
      b=01000110001011011111011110001100
      c=1 z=01000110001011011111011110001100
PASS: a=11100011001101110010010011000110
      b=11100010111101111000010011000101
      c=0 z=11100011001101110010010011000110
PASS: a=01110010101011111111011111100101
      b=10111011110100100111001001110111
      c=0 z=01110010101011111111011111100101
PASS: a=01000111111011001101101110001111
      b=01111001001100000110100111110010
      c=0 z=01000111111011001101101110001111
PASS: a=11110100000000000111101011101000
      b=11100010110010100100111011000101
      c=0 z=11110100000000000111101011101000
PASS: a=11011110100011100010100010111101
      b=10010110101010110101100000101101
      c=1 z=10010110101010110101100000101101
\end{verbatim}

\item[18. ] \verb$(10)$ is the number of times the \verb$repeat$ block is executed.
After changing it to \verb$(500)$ and modifying the code to display only if a
test fails, there is no output, which means all the tests have passed.

\end{enumerate}

\pagebreak

\subsection{LabL4}
\begin{enumerate}
\item[21. ] $ $

\begin{circuitikz}[circuit logic IEC] 
\node[and gate,inputs={nn},and gate IEC symbol={},text height=1cm,] (A) {};
\node[and gate,inputs={nn},and gate IEC symbol={},text height=1cm,] (B) at (0,2) {};
\node[and gate,inputs={nn},and gate IEC symbol={},text height=3cm,] (C) at (2.5,1) {};
\draw  ([xshift=-10pt]A.input 1) node[left] {$a_{2}$} -- (A.input 1);
\draw  ([xshift=-10pt]A.input 2) node[left] {$a_{3}$} -- (A.input 2);
\draw  ([xshift=-10pt]B.input 1) node[left] {$a_{0}$} -- (B.input 1);
\draw  ([xshift=-10pt]B.input 2) node[left] {$a_{1}$} -- (B.input 2);
\draw (A.output) -- ++(1.9,0) node[above left] {$zHi$};
\draw (A.south) -- ++(-90:0.3) |- ++(-15pt,0) node[left] {$c[0]$};
\draw (B.output) -- ++(1.9,0) node[below left] {$zLo$};
\draw (B.south) -- ++(-90:0.3) |- ++(-15pt,0) node[left] {$c[0]$};
\draw (C.output) -- ++(10pt,0) node[right] {$z$};
\draw (C.south) -- ++(-90:0.3) |- ++(-15pt,0) node[left] {$c[1]$};
\end{circuitikz}

Let's verify that this circuit is a 4-to-1 mux:
\begin{itemize}
\item c=0: \linebreak
\verb$c[0]=0$ $\rightarrow$ $zLo=a_0$, $zHi=a_2$ \, \verb$c[1]=0$ $\rightarrow$ $z=zLo=a_0$
\item c=1 \linebreak
\verb$c[0]=1$ $\rightarrow$ $zLo=a_1$, $zHi=a_3$ \, \verb$c[1]=0$ $\rightarrow$ $z=zLo=a_1$
\item c=2 \linebreak
\verb$c[0]=0$ $\rightarrow$ $zLo=a_0$, $zHi=a_2$ \, \verb$c[1]=1$ $\rightarrow$ $z=zHi=a_2$
\item c=3 \linebreak
\verb$c[0]=1$ $\rightarrow$ $zLo=a_1$, $zHi=a_3$ \, \verb$c[1]=1$ $\rightarrow$ $z=zHi=a_3$
\end{itemize}

This is indeed a 4-to-1 multiplexer.

\item[22. ] \verb$yMux4to1$ was tested with several thousand values each
checked against their respective oracle (all done in \verb$LabL4.v$) and
all the tests passed. Therefore the four-way mux behaves as expected.

\end{enumerate}

\subsection{LabL5}
\begin{enumerate}
\item[25. ] \verb$LabL5.v$ was created to perform exhaustive testing
on \verb$yAdder1$ and output any errors (i.e. if MSb of \verb$a+b+cin$
is not equal to \verb$cout$ or if the LSb is different than \verb$z$).
The output was empty (all tests passed).
\end{enumerate}

\subsection{LabL6}
\begin{enumerate}
\item[26. ] To build a 32-bit full adder, 32 (1-bit) full adders are cascaded,
each cell adds the corresponding bits of \verb$a$ and \verb$b$ and the
carry out of each cell is directed into the carry in of the higher order cell.

\item[28. ] In the case of multiplexers, the internal muxes function
independently, each selecting between two inputs and they all work in parallel.
Whereas with full adders, the result of each addition depends on the previous
stage, since the carries are cascaded and each addition has to be performed to
be able to do the next stage addition (pseudo-parallel).

\item[29. ] Here's a sample output of \verb$LabL6.v$ (with a repeat block that
repeats 3 times) testing the 32-bit full adder:
\begin{verbatim}
PASS: a=00010010000101010011010100100100
      b=11000000100010010101111010000001
    cin=0
      z=11010010100111101001001110100101
 expect=11010010100111101001001110100101
PASS: a=10000100100001001101011000001001
      b=10110001111100000101011001100011
    cin=0
      z=00110110011101010010110001101100
 expect=00110110011101010010110001101100
PASS: a=00000110101110010111101100001101
      b=01000110110111111001100110001101
    cin=0
      z=01001101100110010001010010011010
 expect=01001101100110010001010010011010
\end{verbatim}
  All tests pass. The repeat block can easily be adjusted to perform more tests:
a repeat block doing 10000 iterations was also tested, passing all the tests.
Therefore the adder generates correct results.
\end{enumerate}

\subsection{LabL7}
\begin{enumerate}
\item[33. ] After adding \verb$signed$ keyword after each \verb$reg$ and
\verb$wire$ declaration, the tests are still passing. This confirms that
the correctness of the adder is independent of interpretation.
\end{enumerate}

\subsection{LabL8}
\begin{enumerate}
\item[34. ] Subtraction can be done with one adder: if we set \verb$cin=1$,
we're telling the adder that we have a carry (of 1), so in effect, we're adding
one to the number. Now, we just have to \verb$NOT$ the second number and add it
to the first one. This is how we do a subtraction in 2's Complement. If we use
this technique along with a mux (to decide whether we should \verb$NOT$ the
second number or not), we can easily have an arithmetic circuit that takes
two numbers and a control bit, and will subtract the numbers if the \verb$ctrl$
bit is 1 and otherwise add them.

\pagebreak

\item[39. ]
\begin{verbatim}
PASS: a=  303379748
      b=-1064739199
   ctrl=1
      z= 1368118947
 expect= 1368118947
PASS: a=-1309649309
      b=  112818957
   ctrl=1
      z=-1422468266
 expect=-1422468266
PASS: a=-1295874971
      b=-1992863214
   ctrl=1
      z=  696988243
 expect=  696988243
PASS: a=  114806029
      b=  992211318
   ctrl=1
      z= -877405289
 expect= -877405289
PASS: a= 1993627629
      b= 1177417612
   ctrl=1
      z=  816210017
 expect=  816210017
PASS: a= -482925370
      b= -487095099
   ctrl=0
      z= -970020469
 expect= -970020469
\end{verbatim}
  \verb$yArith$ works as expected, as demonstrated with the above random test cases.
\end{enumerate}

\pagebreak

\subsection{LabL9}
\begin{enumerate}
\item[44. ] The ALU works correctly and as expected on all four operations.
Here's some sample output (2 tests for each operation):
\begin{verbatim}
$ vvp a.out +op=0
PASS:  a=00010010000101010011010100100100
       b=11000000100010010101111010000001
     a&b=00000000000000010001010000000000

PASS:  a=10000100100001001101011000001001
       b=10110001111100000101011001100011
     a&b=10000000100000000101011000000001


$ vvp a.out +op=1
PASS:  a=00010010000101010011010100100100
       b=11000000100010010101111010000001
     a|b=11010010100111010111111110100101

PASS:  a=10000100100001001101011000001001
       b=10110001111100000101011001100011
     a|b=10110101111101001101011001101011


$ vvp a.out +op=2
PASS:  a=00010010000101010011010100100100
       b=11000000100010010101111010000001
     a+b=11010010100111101001001110100101

PASS:  a=10000100100001001101011000001001
       b=10110001111100000101011001100011
     a+b=00110110011101010010110001101100


$ vvp a.out +op=6
PASS:  a=00010010000101010011010100100100
       b=11000000100010010101111010000001
     a-b=01010001100010111101011010100011

PASS:  a=10000100100001001101011000001001
       b=10110001111100000101011001100011
     a-b=11010010100101000111111110100110
\end{verbatim}
  Note: \verb#$# is the shell prompt.
\end{enumerate}

\subsection{LabL10}
\begin{enumerate}
\item[51. ] Sample output for the \verb$slt$ operation, all tests passing:
\begin{verbatim}
$ vvp a.out +op=7
PASS:  a=00010010000101010011010100100100
       b=11000000100010010101111010000001
     a<b=00000000000000000000000000000000

PASS:  a=10000100100001001101011000001001
       b=10110001111100000101011001100011
     a<b=00000000000000000000000000000001

PASS:  a=00000110101110010111101100001101
       b=01000110110111111001100110001101
     a<b=00000000000000000000000000000001

PASS:  a=10110010110000101000010001100101
       b=10001001001101110101001000010010
     a<b=00000000000000000000000000000000

PASS:  a=00000000111100111110001100000001
       b=00000110110101111100110100001101
     a<b=00000000000000000000000000000001
\end{verbatim}
\end{enumerate}

\subsection{LabL11}
\begin{enumerate}
\item[52. ] This \textit{algorithm} is basically the \verb$NOR$ gate, and it
only produces 1 when all inputs are 0. Let's take a look at the truth table
of \verb$NOR$:
\begin{displaymath}
\begin{array}{|c|c|c|}
\hline
   P
 & Q
 & P \downarrow Q \\
\hline
0 & 0 & 1 \\
0 & 1 & 0 \\
1 & 0 & 0 \\
1 & 1 & 0 \\
\hline
\end{array}
\end{displaymath}
The only difference between this table and our algorithm is that we have
32 inputs instead of 2. The truth table for 32 inputs will be \textit{huge}
but again the only case the result of the function is 1 will be the case where
all 32 values are 0; because even if one value is 1, the result of \verb$OR$
will be 1, and \verb$NOT$ of that will be 0.

\pagebreak

\item[55. ] Here are some sample outputs after implementing the
\verb$zero$ (zero output detection):
\begin{verbatim}
$ vvp a.out +op=7
PASS:  a=00010010000101010011010100100100
       b=11000000100010010101111010000001
     a<b=00000000000000000000000000000000
    zero=1

PASS:  a=10110001111100000101011001100011
       b=00000110101110010111101100001101
     a<b=00000000000000000000000000000001
    zero=0

PASS:  a=10110010110000101000010001100101
       b=10001001001101110101001000010010
     a<b=00000000000000000000000000000000
    zero=1
\end{verbatim}
\end{enumerate}

\subsection{CPU}
\begin{enumerate}
\item[57. ] \verb$cpu.z$ was tested with existing modules (e.g. \verb$LabL9.v$,
\verb$LabL11.v$, etc) and it works correctly.
\end{enumerate}

%----------------------------------------------------------------------------------------
%	SECTION 4
%----------------------------------------------------------------------------------------

\section{Evaluations and Conclusions}

Since I consider myself to be mostly a \textit{software guy}, doing the
exercises and following the manual of this lab was a new and pleasant
experience; it's very interesting to go further down towards the hardware
and to learn how simple [combinational] circuits work together to make up
various processing units (such as ALU). And to be able to design a CPU
(although very limited and perhaps overly simplified), is very exciting.\newline

I think at one point or another we've all been told that
\textit{computers only know 0's and 1's}, and although at heart we know that's
true, nowadays that \verb$C$ is considered a low-level language, imagining
\textit{how} that is might not be very easy, and that's why this course is
so interesting to me because it takes me on a journey to lower levels and areas
that I've always taken as granted, and asks me to implement them!

\pagebreak

%----------------------------------------------------------------------------------------
%	SECTION 5
%----------------------------------------------------------------------------------------

\section{Source Codes}


\subsection{LabL1}
\lstinputlisting[language=verilog]{yMux1.v}
\lstinputlisting[language=verilog]{LabL1.v}

\subsection{LabL2}
\lstinputlisting[language=verilog]{yMux2.v}
\lstinputlisting[language=verilog]{LabL2.v}

\subsection{LabL3}
\lstinputlisting[language=verilog]{yMux.v}
\lstinputlisting[language=verilog]{LabL3.v}

\subsection{LabL4}
\lstinputlisting[language=verilog]{yMux4to1.v}
\lstinputlisting[language=verilog]{LabL4.v}

\subsection{LabL5}
\lstinputlisting[language=verilog]{yAdder1.v}
\lstinputlisting[language=verilog]{LabL5.v}

\subsection{LabL6}
\lstinputlisting[language=verilog]{yAdder.v}
\lstinputlisting[language=verilog]{LabL6.v}

\pagebreak

\subsection{LabL7}
\lstinputlisting[language=verilog]{LabL7.v}

\subsection{LabL8}
\lstinputlisting[language=verilog]{yArith.v}
\lstinputlisting[language=verilog]{LabL8.v}

\subsection{LabL9}
\verb$yAlu.v$: See \textbf{LabL11} sources below (completely implemented)

\verb$LabL9.v$:
\lstinputlisting[language=verilog]{LabL9.v}

\subsection{LabL10}
\verb$yAlu.v$: See \textbf{LabL11} sources below (completely implemented)

\verb$LabL10.v$:
\lstinputlisting[language=verilog]{LabL10.v}

\subsection{LabL11}
\lstinputlisting[language=verilog]{yAlu.v}
\lstinputlisting[language=verilog]{LabL11.v}

\pagebreak

\subsection{CPU}
\lstinputlisting[language=verilog]{cpu.v}

%----------------------------------------------------------------------------------------

\end{document}