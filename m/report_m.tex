%----------------------------------------------------------------------------------------
%	PACKAGES AND DOCUMENT CONFIGURATIONS
%----------------------------------------------------------------------------------------

\documentclass{article}

\usepackage{natbib} % Required to change bibliography style to APA
\usepackage{amsmath} % Required for some math elements 
\usepackage{listings}
\usepackage{color}

\definecolor{dkgreen}{rgb}{0,0.6,0}
\definecolor{gray}{rgb}{0.5,0.5,0.5}
\definecolor{mauve}{rgb}{0.58,0,0.82}

\lstset{frame=tb,
  language=Verilog,
  aboveskip=3mm,
  belowskip=3mm,
  showstringspaces=false,
  columns=flexible,
  basicstyle={\small\ttfamily},
  numbers=none,
  numberstyle=\tiny\color{gray},
  captionpos=b,                   % sets the caption-position to bottom
  title=\lstname,                 % show the filename of files included with \lstinputlisting;
  keywordstyle=\color{blue},
  commentstyle=\color{dkgreen},
  stringstyle=\color{mauve},
  breaklines=true,
  breakatwhitespace=true,
  tabsize=4
}

\newenvironment{statement}{\par\vspace{50ex}}{\clearpage}

\setlength\parindent{0pt} % Removes all indentation from paragraphs

% Make numbering in the enumerate environment by letter rather than number (e.g. section 6)
\renewcommand{\labelenumi}{\alph{enumi}.}

%----------------------------------------------------------------------------------------
%	DOCUMENT INFORMATION
%----------------------------------------------------------------------------------------

\title{\textsc{Lab M} \\ Building the CPU } % Title

\author{\textsc{Mohammadamin Bandali}} % Author name

\date{} % no date

\begin{document}

\maketitle % Insert the title, author and date

\begin{center}
\begin{tabular}{l r}
Student Number: & XXXXXXXXX \\ 
Date Performed: & April 7, 2015 \\ % Date the experiment was performed
Lab Location: & Lassonde 1006 \\ 
Course Name: & Computer Organization \\ 
Course Code: & EECS 2021 \\ 
Course Section: & Z, Lab-02\\ 
Instructor: & Professor Peter Lian % Instructor/supervisor
\end{tabular}
\end{center}

\begin{statement}
“The work in this report is my own. I have read and understood York University
academic honesty guidelines and I did not violate Senate Policy on Academic
Honesty in writing this report.”
\end{statement}

%----------------------------------------------------------------------------------------
%	SECTION 1
%----------------------------------------------------------------------------------------

\section{Introduction}

% TODO

%----------------------------------------------------------------------------------------
%	SECTION 2
%----------------------------------------------------------------------------------------

\section{Methods and Equipment}

\begin{center}
\begin{tabular}{l r}
Computer: & MacBook Air 13-inch, Mid 2013 \\
CPU: & 1.7GHz dual-core Intel Core i7 \\
RAM: & 8GB 1600MHz LPDDR3 \\
Operating System: & Arch GNU/Linux \\
Kernel Version: & 3.17.6
\end{tabular}
\end{center}

%----------------------------------------------------------------------------------------
%	SECTION 3
%----------------------------------------------------------------------------------------

\section{Experimental Procedure and Results}

\subsection{LabM1}
\begin{enumerate}
\item[4. ] The first \verb#$display# statement will show an unknown value
(i.e. \verb$x$) for \verb$z$, because the clock is 0 and the register hasn't
stored the value of \verb$d$. In the second output, \verb$z$ will be set to
\verb$d$, ($= 20$) since the clock is 1 and the register changes on the rising
edge. The third output, \verb$z$ is still 20, since it's the falling edge of the
clock. Finally, the last output will be a new value of z ($= 30$) since the clock
is on the rising edge.

\item[5. ] The register works as expected, ignoring the input when \verb$enabled=0$
and storing them (on the rising edges of the clock) when \verb$enabled=1$:
\begin{verbatim}
$ vvp a.out +enable=0
clk=0 d=        15, z=         x
clk=1 d=        20, z=         x
clk=0 d=        25, z=         x
clk=1 d=        30, z=         x

$ vvp a.out +enable=1
clk=0 d=        15, z=         x
clk=1 d=        20, z=        20
clk=0 d=        25, z=        20
clk=1 d=        30, z=        30
\end{verbatim}

\end{enumerate}

%----------------------------------------------------------------------------------------
%	SECTION 4
%----------------------------------------------------------------------------------------

\section{Evaluations and Conclusions}

% TODO

%----------------------------------------------------------------------------------------
%	SECTION 5
%----------------------------------------------------------------------------------------

\section{Source Codes}

\subsection{LabM1}
\lstinputlisting[language=verilog]{LabM1.v}

%----------------------------------------------------------------------------------------

\end{document}