%----------------------------------------------------------------------------------------
%	PACKAGES AND DOCUMENT CONFIGURATIONS
%----------------------------------------------------------------------------------------

\documentclass{article}

\usepackage{natbib} % Required to change bibliography style to APA
\usepackage{amsmath} % Required for some math elements 
\usepackage{listings}
\usepackage{color}
\usepackage{../mips}

\definecolor{dkgreen}{rgb}{0,0.6,0}
\definecolor{gray}{rgb}{0.5,0.5,0.5}
\definecolor{mauve}{rgb}{0.58,0,0.82}

\lstset{frame=tb,
  language=Verilog,
  aboveskip=3mm,
  belowskip=3mm,
  showstringspaces=false,
  columns=flexible,
  basicstyle={\small\ttfamily},
  numbers=none,
  numberstyle=\tiny\color{gray},
  captionpos=b,                   % sets the caption-position to bottom
  title=\lstname,                 % show the filename of files included with \lstinputlisting;
  keywordstyle=\color{blue},
  commentstyle=\color{dkgreen},
  stringstyle=\color{mauve},
  breaklines=true,
  breakatwhitespace=true,
  tabsize=4
}

\lstdefinestyle{Assembly}{ %
  language=[mips]Assembler,       % the language of the code
  basicstyle={\small\ttfamily},       % the size of the fonts that are used for the code
  numbers=left,                   % where to put the line-numbers
  numberstyle=\footnotesize\tiny\color{gray},  % the style that is used for the line-numbers
  stepnumber=1,                   % the step between two line-numbers. If it's 1, each line 
                                  % will be numbered
  numbersep=5pt,                  % how far the line-numbers are from the code
  backgroundcolor=\color{white},  % choose the background color. You must add \usepackage{color}
  showspaces=false,               % show spaces adding particular underscores
  showstringspaces=false,         % underline spaces within strings
  showtabs=false,                 % show tabs within strings adding particular underscores
  frame=tb,                       % adds a frame around the code
  rulecolor=\color{black},        % if not set, the frame-color may be changed on line-breaks within not-black text (e.g. commens (green here))
  tabsize=4,                      % sets default tabsize to 2 spaces
  captionpos=b,                   % sets the caption-position to bottom
  breaklines=true,                % sets automatic line breaking
  breakatwhitespace=false,        % sets if automatic breaks should only happen at whitespace
  title=\lstname,                 % show the filename of files included with \lstinputlisting;
                                  % also try caption instead of title
  keywordstyle=\color{blue},          % keyword style
  commentstyle=\color{dkgreen},       % comment style
  stringstyle=\color{mauve},         % string literal style
  escapeinside={\%*}{*)},            % if you want to add a comment within your code
  morekeywords={*,...}               % if you want to add more keywords to the set
}

\newenvironment{statement}{\par\vspace{50ex}}{\clearpage}

\setlength\parindent{0pt} % Removes all indentation from paragraphs

% Make numbering in the enumerate environment by letter rather than number (e.g. section 6)
\renewcommand{\labelenumi}{\alph{enumi}.}

%----------------------------------------------------------------------------------------
%	DOCUMENT INFORMATION
%----------------------------------------------------------------------------------------

\title{\textsc{Lab M} \\ Building the CPU } % Title

\author{\textsc{Mohammadamin Bandali}} % Author name

\date{} % no date

\begin{document}

\maketitle % Insert the title, author and date

\begin{center}
\begin{tabular}{l r}
Student Number: & XXXXXXXXX \\ 
Date Performed: & April 7, 2015 \\ % Date the experiment was performed
Lab Location: & Lassonde 1006 \\ 
Course Name: & Computer Organization \\ 
Course Code: & EECS 2021 \\ 
Course Section: & Z, Lab-02\\ 
Instructor: & Professor Peter Lian % Instructor/supervisor
\end{tabular}
\end{center}

\begin{statement}
“The work in this report is my own. I have read and understood York University
academic honesty guidelines and I did not violate Senate Policy on Academic
Honesty in writing this report.”
\end{statement}

%----------------------------------------------------------------------------------------
%	SECTION 1
%----------------------------------------------------------------------------------------

\section{Introduction}

% TODO

%----------------------------------------------------------------------------------------
%	SECTION 2
%----------------------------------------------------------------------------------------

\section{Methods and Equipment}

\begin{center}
\begin{tabular}{l r}
Computer: & MacBook Air 13-inch, Mid 2013 \\
CPU: & 1.7GHz dual-core Intel Core i7 \\
RAM: & 8GB 1600MHz LPDDR3 \\
Operating System: & Arch GNU/Linux \\
Kernel Version: & 3.17.6
\end{tabular}
\end{center}

%----------------------------------------------------------------------------------------
%	SECTION 3
%----------------------------------------------------------------------------------------

\section{Experimental Procedure and Results}

\subsection{LabM1}
\begin{enumerate}
\item[4. ] The first \verb#$display# statement will show an unknown value
(i.e. \verb$x$) for \verb$z$, because the clock is 0 and the register hasn't
stored the value of \verb$d$. In the second output, \verb$z$ will be set to
\verb$d$, ($= 20$) since the clock is 1 and the register changes on the rising
edge. The third output, \verb$z$ is still 20, since it's the falling edge of the
clock. Finally, the last output will be a new value of z ($= 30$) since the clock
is on the rising edge.

\item[5. ] The register works as expected, ignoring the input when \verb$enabled=0$
and storing them (on the rising edges of the clock) when \verb$enabled=1$:
\begin{verbatim}
$ vvp a.out +enable=0
clk=0 d=        15, z=         x
clk=1 d=        20, z=         x
clk=0 d=        25, z=         x
clk=1 d=        30, z=         x

$ vvp a.out +enable=1
clk=0 d=        15, z=         x
clk=1 d=        20, z=        20
clk=0 d=        25, z=        20
clk=1 d=        30, z=        30
\end{verbatim}
\end{enumerate}

\subsection{LabM2}
\begin{enumerate}
\item[10. ] The register behaves as expected: Initially, the value is unknown,
then at each rising edge the value is updated to the value of input \verb$d$.
\begin{verbatim}
$ vvp a.out +enable=1
    0: clk=0,d=         x,z=         x,expect=         x
    2: clk=0,d= 303379748,z=         x,expect=         x
    4: clk=0,d=3230228097,z=         x,expect=         x
    5: clk=1,d=3230228097,z=3230228097,expect=3230228097
    6: clk=1,d=2223298057,z=3230228097,expect=3230228097
    8: clk=1,d=2985317987,z=3230228097,expect=3230228097
   10: clk=0,d= 112818957,z=3230228097,expect=3230228097
   12: clk=0,d=1189058957,z=3230228097,expect=3230228097
   14: clk=0,d=2999092325,z=3230228097,expect=3230228097
   15: clk=1,d=2999092325,z=2999092325,expect=2999092325
   16: clk=1,d=2302104082,z=2999092325,expect=2999092325
   18: clk=1,d=  15983361,z=2999092325,expect=2999092325
   20: clk=0,d= 114806029,z=2999092325,expect=2999092325
   22: clk=0,d= 992211318,z=2999092325,expect=2999092325
   24: clk=0,d= 512609597,z=2999092325,expect=2999092325
   25: clk=1,d= 512609597,z= 512609597,expect= 512609597
   26: clk=1,d=1993627629,z= 512609597,expect= 512609597
   28: clk=1,d=1177417612,z= 512609597,expect= 512609597
   30: clk=0,d=2097015289,z= 512609597,expect= 512609597
   32: clk=0,d=3812041926,z= 512609597,expect= 512609597
   34: clk=0,d=3807872197,z= 512609597,expect= 512609597
   35: clk=1,d=3807872197,z=3807872197,expect=3807872197
   36: clk=1,d=3574846122,z=3807872197,expect=3807872197
   38: clk=1,d=1924134885,z=3807872197,expect=3807872197
   40: clk=0,d=3151131255,z=3807872197,expect=3807872197
\end{verbatim}
\end{enumerate}

\subsection{LabM3}
\begin{enumerate}
\item[14. ] The output of the program with \verb$w$ set:
\begin{verbatim}
$ vvp a.out +w=1
rn1= 4,rd1=        16  rn2= 1,rd2=         1
rn1= 9,rd1=        81  rn2= 3,rd2=         9
rn1=13,rd1=       169  rn2=13,rd2=       169
rn1= 5,rd1=        25  rn2=18,rd2=       324
rn1= 1,rd1=         1  rn2=13,rd2=       169
rn1=22,rd1=       484  rn2=29,rd2=       841
rn1=13,rd1=       169  rn2=12,rd2=       144
rn1=25,rd1=       625  rn2= 6,rd2=        36
rn1= 5,rd1=        25  rn2=10,rd2=       100
rn1= 5,rd1=        25  rn2=23,rd2=       529
\end{verbatim}
  We can see that each register value is the square of its register
number, as we intended. Therefore the program is working correctly.

\item[15. ] With \verb$w$ cleared, nothing is stored in the registers,
therefore the value of each one will be unknown (\verb$x$):
\begin{verbatim}
$ vvp a.out +w=0
rn1= 4,rd1=         x  rn2= 1,rd2=         x
rn1= 9,rd1=         x  rn2= 3,rd2=         x
rn1=13,rd1=         x  rn2=13,rd2=         x
rn1= 5,rd1=         x  rn2=18,rd2=         x
rn1= 1,rd1=         x  rn2=13,rd2=         x
rn1=22,rd1=         x  rn2=29,rd2=         x
rn1=13,rd1=         x  rn2=12,rd2=         x
rn1=25,rd1=         x  rn2= 6,rd2=         x
rn1= 5,rd1=         x  rn2=10,rd2=         x
rn1= 5,rd1=         x  rn2=23,rd2=         x
\end{verbatim}
The output verifies that the register file behaves as expected.
\end{enumerate}

\subsection{LabM4}
\begin{enumerate}
\item[20. ] The content of the memory at address 16 should be 12345678 (hex),
at 20 should be unknown (since we didn't write to that address);
and at 24 should be 89abcdef (hex).

\item[21. ] The output matches our prediction:
\begin{verbatim}
$ vvp a.out
ERROR: ../hrLib/mem.v:22: $readmemh: Unable to open ram.dat
 for reading.
Address         16 contains 12345678
Address         20 contains xxxxxxxx
Address         24 contains 89abcdef
\end{verbatim}
  Therefore the program behaves as expected.

\item[22. ] According to the lab manual (see '$17$ above') attempting to
read / write using a non-word address should display an "unaligned address"
error; but this was not the case. \newline

Looking at the source code for the memory module on a lab machine at
\begin{verbatim}
/cs/fac/pkg/verimips/hrLib/mem.v
\end{verbatim}
shows that the corresponding lines (line $29$ for \verb$read$ and
$50$ for \verb$write$) for displaying the error were commented out. \newline

So, no error is displayed when using an unaligned address, but here's what
happens instead:

\begin{itemize}
\item read: \verb$memOut$ is set to \verb$32'hxxxxxxxx$ (a 32-bit \textit{unknown}).
\item write: no action is taken.
\end{itemize}
\end{enumerate}

\subsection{LabM5}
\begin{enumerate}
\item[28. ] LabM5 outputs the program (\verb$ram.dat$) in machine language:
\begin{verbatim}
Address 00000080 contains 00006820
Address 00000084 contains 00008020
Address 00000088 contains 00002020
Address 0000008c contains 8da80050
Address 00000090 contains 11000004
Address 00000094 contains 02088020
Address 00000098 contains 00882025
Address 0000009c contains 21ad0004
Address 000000a0 contains 08000023
Address 000000a4 contains ac100020
Address 000000a8 contains ac040024
\end{verbatim}
\end{enumerate}

\subsection{LabM6}
\begin{enumerate}
\item[29. ] LabM6 produces the correct output, the memory content in an
instruction-aware format:
\begin{verbatim}
0 0 0 13 32
0 0 0 16 32
0 0 0 4 32
35 13 8 80
4 8 0 4
0 16 8 16 32
0 4 8 4 37
8 13 13 4
2 35
43 0 16 32
43 0 4 36
\end{verbatim}
\end{enumerate}

\subsection{LabM7}
\begin{enumerate}
\item[35. ] The output is the instructions in our \verb$ram.dat$ file:
\begin{verbatim}
instruction = 00006820
instruction = 00008020
instruction = 00002020
instruction = 8da80050
instruction = 11000004
instruction = 02088020
instruction = 00882025
instruction = 21ad0004
instruction = 08000023
instruction = ac100020
instruction = ac040024
\end{verbatim}
\end{enumerate}

\subsection{LabM8}
\begin{enumerate}
\item[50. ] The output of LabM8:
\begin{verbatim}
00006820: rd1= 0 rd2= 0 imm=00006820 jTarget=0006820 z= 0 zero=1
00008020: rd1= 0 rd2= 0 imm=ffff8020 jTarget=0008020 z= 0 zero=1
00002020: rd1= 0 rd2= 0 imm=00002020 jTarget=0002020 z= 0 zero=1
8da80050: rd1= 0 rd2= x imm=00000050 jTarget=1a80050 z=80 zero=0
11000004: rd1=80 rd2= 0 imm=00000004 jTarget=1000004 z=84 zero=0
02088020: rd1= 0 rd2=80 imm=ffff8020 jTarget=2088020 z=80 zero=0
00882025: rd1= 0 rd2=80 imm=00002025 jTarget=0882025 z=80 zero=0
21ad0004: rd1= 0 rd2= 0 imm=00000004 jTarget=1ad0004 z= 4 zero=0
08000023: rd1= 0 rd2= 0 imm=00000023 jTarget=0000023 z=35 zero=0
ac100020: rd1= 0 rd2=80 imm=00000020 jTarget=0100020 z=32 zero=0
ac040024: rd1= 0 rd2=80 imm=00000024 jTarget=0040024 z=36 zero=0
\end{verbatim}
  Looking at each instruction they represent (in order): add, add, add, lw, beq,
add, or, addi, j, sw, sw. We can see that each line is consistent with our
datapath (looking at \verb$rd1$, \verb$rd2$, \verb$imm$ and \verb$z$).
\end{enumerate}

\subsection{LabM9}
\begin{enumerate}
\item[57. ] The output of LabM9 matches the given output in the lab manual:
\begin{verbatim}
00006820: rd1= 0 rd2= 0 z=  0 zero=1 wb= 0
00008020: rd1= 0 rd2= 0 z=  0 zero=1 wb= 0
00002020: rd1= 0 rd2= 0 z=  0 zero=1 wb= 0
8da80050: rd1= 0 rd2= x z= 80 zero=0 wb= 1
11000004: rd1= 1 rd2= 0 z=  1 zero=0 wb= 1
02088020: rd1= 0 rd2= 1 z=  1 zero=0 wb= 1
00882025: rd1= 0 rd2= 1 z=  1 zero=0 wb= 1
21ad0004: rd1= 0 rd2= 0 z=  4 zero=0 wb= 4
08000023: rd1= 0 rd2= 0 z= 35 zero=0 wb=35
ac100020: rd1= 0 rd2= 1 z= 32 zero=0 wb=32
ac040024: rd1= 0 rd2= 1 z= 36 zero=0 wb=36
\end{verbatim}
\end{enumerate}

%----------------------------------------------------------------------------------------
%	SECTION 4
%----------------------------------------------------------------------------------------

\section{Evaluations and Conclusions}

% TODO

%----------------------------------------------------------------------------------------
%	SECTION 5
%----------------------------------------------------------------------------------------

\section{Source Codes}

\subsection{LabM1}
\lstinputlisting[language=verilog]{LabM1.v}

\subsection{LabM2}
\lstinputlisting[language=verilog]{LabM2.v}

\subsection{LabM3}
\lstinputlisting[language=verilog]{LabM3.v}

\subsection{LabM4}
\lstinputlisting[language=verilog]{LabM4.v}

\subsection{LabM5}
\lstinputlisting[language=verilog]{LabM5.v}
\lstinputlisting[style=assembly]{LabM5.s}
LabM5.s was loaded into spim to generate the machine encoding of each
statement, used for writing the following \verb$ram.dat$ program:
\lstinputlisting{ram.dat}

\subsection{LabM6}
\lstinputlisting[language=verilog]{LabM6.v}

\subsection{LabM7}
\lstinputlisting[language=verilog]{LabM7.v}

\subsection{LabM8}
\lstinputlisting[language=verilog]{LabM8.v}

\subsection{LabM9}
\lstinputlisting[language=verilog]{LabM9.v}

\subsection{CPU}
\lstinputlisting[language=verilog]{cpu.v}

%----------------------------------------------------------------------------------------

\end{document}
